\documentclass[12pt, nofootinbib, notitlepage]{revtex4}
%~ \documentclass{article}
\usepackage{amsmath}
%~ \usepackage[margin=3.4cm]{geometry}
\usepackage{graphicx}
\usepackage{mathdots} % for \iddots
\pagenumbering{gobble}
\usepackage{revsymb} % openone
\usepackage{cancel}
\usepackage{units}

% A bunch of useful commands I keep creating in all of LaTeX files
% Copyright(C) 2018 by Keith.David.Pedersen@gmail.com (hepguy.com)

% The dummy super-script, because $x_a^{}$ looks better than $x_a$
% due to a lower-riding subscript. 
\newcommand{\ds}{^{}}
% This example demonstrates how \ds makes subscripts match up better when the symbol is squared
\newcommand{\dsExample}{$\overset{\text{this one is {\tt \$x\_a{\textbackslash}ds\$}}}{
x_a		\;
x_a^2		\;\overset{\downarrow}{
x_a\ds	}}$\hrule}

% If not already defined (by \usepackage{amsmath}), macro for text in math mode
\ifcsname text\endcsname%
\else
\newcommand{\text}[1]{\mathrm{#1}}
\fi

\newcommand{\I}{i} % sqrt{-1}
\newcommand{\abs}[1]{\left|#1\right|} % absolute value
\newcommand{\bigO}[1]{\mathcal{O}(#1)} % big O notation
\newcommand{\diff}[1]{\text{d}#1} % d/dx => \diff{} / \diff{x}, because d shouldn't be italicized
\newcommand{\atanh}{\text{arctanh}}
\newcommand{\ExAngle}[1]{\langle#1\rangle} % expectation values
\newcommand{\Ex}[1]{\text{Ex}\left(#1\right)} % expectation values
\newcommand{\Exx}[1]{\text{Ex}(#1)} % expectation values, without \left \right

% The manually defined ident will go bold, \openone will not
\newcommand{\ident}{\text{{\small $1$}}\hspace{-0.37em}1}
%~ \ifcsname openone\endcsname
	%~ \newcommand{\ident}{\openone}
%~ \else
	%~ \newcommand{\ident}{\text{{\small $1$}}\hspace{-0.37em}1}
%~ \fi

% Bra-ket notation
\newcommand{\bra}[1]{\left\langle #1\right|}
\newcommand{\ket}[1]{\left|#1\right\rangle}
\newcommand{\braket}[2]{\left\langle#1|#2\right\rangle}

% If not already defined (by \usepackage{units}), define unit (e.g. $3\,\unit{m}$) 
\ifcsname unit\endcsname%
\else
\newcommand{\unit}[2][]{#1\,\text{#2}}
\newcommand{\unitfrac}[3][]{#1\,\nicefrac{\text{#2}}{\text{#3}}}
\fi

% If not already defined (by \usepackage{cancel}), define cancel (lower-left to upper-right strikethrough)
\ifcsname cancel\endcsname%
\else
\newcommand{\cancel}[1]{\not\mathrel{#1}}
\fi

\newcommand{\angstrom}{\aa ngstr\"om}

% Print 10 paragraphs of dummy text, to make formatting decisions.
\newcommand{\lipsum}{Lorem ipsum dolor sit amet, consectetur adipiscing elit. In ac arcu at erat fermentum ultrices. Fusce id justo quis massa vestibulum venenatis posuere id enim. Etiam diam augue, dapibus eget facilisis vitae, accumsan non ante. Fusce lorem ipsum, dignissim facilisis pretium lobortis, consectetur at nibh. In id est vel dolor porta varius. Curabitur ultrices tristique tempor. Phasellus semper cursus tincidunt. Morbi consequat mauris nec velit convallis, ac volutpat enim venenatis. Fusce a erat sed sapien rhoncus luctus in nec arcu. Cras metus erat, cursus sed justo nec, lacinia bibendum nunc.
\par Aenean non mollis leo. Morbi faucibus viverra quam ut sollicitudin. Duis fringilla tempor ante eget dapibus. Donec at ex consectetur, hendrerit eros quis, efficitur diam. Sed hendrerit volutpat neque euismod faucibus. Curabitur est orci, molestie id faucibus vestibulum, aliquet ut arcu. Nam tristique, leo nec porttitor condimentum, sapien lorem aliquam magna, in tincidunt metus eros at sem. Nam ut malesuada sapien. Donec vitae bibendum sapien. In efficitur porttitor ex nec consequat. Vivamus molestie eu augue non gravida. Curabitur egestas, ligula id imperdiet sagittis, metus felis placerat est, vitae vehicula quam lectus sit amet sem. Ut vel tristique ante. Aliquam erat volutpat.
\par Ut viverra lorem ex, eu consectetur enim aliquam a. Pellentesque tellus lacus, ultricies eget vehicula quis, laoreet sit amet turpis. Nam at massa vitae enim pretium cursus hendrerit pellentesque lorem. In mauris tortor, ornare vitae tellus ac, faucibus sollicitudin nibh. Nulla viverra libero vitae quam porttitor suscipit. Vivamus et condimentum nunc. Vestibulum in nisi nibh. Fusce nisl felis, bibendum id lectus sit amet, ornare gravida sapien. Fusce tempor enim at enim sodales vulputate. Fusce eu lacus faucibus, pharetra ipsum sit amet, maximus enim.
\par Etiam tempus leo lacus, in mattis ante pretium accumsan. Cras dapibus nisi enim, at volutpat sapien feugiat ac. Integer non bibendum felis. Fusce finibus volutpat egestas. Praesent tincidunt pulvinar molestie. Etiam ut egestas dui. Integer sit amet augue eget quam hendrerit tincidunt ut sed dolor. Quisque ornare tellus ac pretium aliquam. Donec orci neque, vulputate at lorem at, laoreet auctor tellus.
\par Vestibulum turpis enim, condimentum ac pulvinar quis, dapibus eu velit. Curabitur sagittis malesuada molestie. Sed sit amet lorem in mauris eleifend lacinia. Mauris scelerisque rutrum orci quis aliquam. Cras lectus orci, cursus sit amet magna ac, bibendum efficitur massa. Nulla nec felis interdum, venenatis massa a, cursus dolor. Morbi molestie risus vitae mauris aliquam, a fermentum justo rhoncus. In sollicitudin rutrum eros feugiat euismod. Cras at aliquet libero.
\par Sed tristique ligula ut quam suscipit, sodales tincidunt quam blandit. Quisque condimentum et quam gravida auctor. Vestibulum facilisis libero eu erat tempus sollicitudin. Phasellus scelerisque mi ac nisi pulvinar, eget mattis lorem luctus. Donec eu egestas felis. Integer pharetra diam metus, et molestie nisi varius a. Suspendisse metus odio, volutpat id porttitor quis, posuere eget ligula. Duis feugiat accumsan eros sed iaculis. Sed ac libero sed metus varius imperdiet nec sed leo. Donec feugiat id lorem in euismod.
\par Donec viverra nibh eu lectus lobortis, ut laoreet metus dignissim. Praesent consequat, ante ac egestas tristique, lectus arcu varius elit, eget luctus odio lectus convallis purus. Quisque egestas est risus, molestie rutrum dui pulvinar vitae. In tortor metus, cursus quis faucibus quis, vulputate non nibh. Duis blandit ligula vitae elit ultrices ultricies. Phasellus lobortis mauris ex, mollis feugiat nisl dictum sit amet. Curabitur aliquam elit elit, at imperdiet metus laoreet eget. Fusce at nibh sed nisl mollis auctor non et elit. Suspendisse a congue massa. Class aptent taciti sociosqu ad litora torquent per conubia nostra, per inceptos himenaeos. Praesent fringilla, eros venenatis vehicula cursus, sapien dolor blandit eros, ac venenatis nulla purus id magna. Lorem ipsum dolor sit amet, consectetur adipiscing elit. Duis eros erat, volutpat in eros sed, tempus malesuada nisl.
\par Nunc tellus leo, tincidunt sollicitudin tincidunt ut, venenatis et ligula. Praesent ac tellus eget est aliquam posuere. Aenean sagittis dictum nisl at malesuada. Duis sollicitudin quam justo, nec semper lorem fermentum eu. Curabitur convallis consectetur massa, non semper odio pellentesque ut. Vestibulum libero diam, feugiat in est id, sodales dapibus ex. Nulla mattis libero dolor, id accumsan lacus ultricies id.
\par Etiam imperdiet enim ut cursus pharetra. Nunc tempus ullamcorper eleifend. Nullam vitae porta nunc. Maecenas sit amet semper dui. Mauris augue metus, rhoncus id imperdiet ut, fringilla ac ligula. Mauris laoreet ac diam ut iaculis. Curabitur quis bibendum elit. Praesent lectus mi, aliquam id consectetur vitae, hendrerit a felis. Proin a gravida lectus, non faucibus ante. Nulla malesuada augue eget ligula fermentum hendrerit. Curabitur accumsan urna id dui efficitur blandit. Integer vitae mi commodo, porttitor mauris a, mattis nisi.
\par Suspendisse nec commodo lorem. Suspendisse varius bibendum tempus. Sed metus purus, ultrices vel dolor at, lobortis auctor ante. Etiam scelerisque lorem ut enim blandit, ut tristique leo faucibus. Mauris tincidunt dolor vel neque porta, eget ultrices metus pulvinar. Sed semper ante non nulla pharetra, at luctus urna varius. Phasellus eu magna sed enim porta mattis ut et sapien. Nam at mattis magna.}

% Print a table which shows all the Greek letters you can use in math mode
\newcommand{\allgreek}{
\newcommand{\blank}{}
\begin{table}
\begin{tabular}{|c|c|c|}
\hline
{\bf name} & {\bf lowercase} & {\bf uppercase} \\ \hline
alpha    & $\alpha$     & \blank       \\ \hline
beta     & $\beta$      & \blank       \\ \hline
gamma    & $\gamma$     & $\Gamma$     \\ \hline
delta    & $\delta$     & $\Delta$     \\ \hline   \hline

epsilon  & $\epsilon$   & \blank       \\ \hline
zeta     & $\zeta$      & \blank       \\ \hline
eta      & $\eta$       & \blank       \\ \hline
theta    & $\theta$     & $\Theta$     \\ \hline   \hline

iota     & $\iota$      & \blank       \\ \hline
kappa    & $\kappa$     & \blank       \\ \hline
lambda   & $\lambda$    & $\Lambda$    \\ \hline
mu       & $\mu$        & \blank       \\ \hline   \hline

nu       & $\nu$        & \blank       \\ \hline
xi       & $\xi$        & $\Xi$        \\ \hline
omicron  & \blank       & \blank       \\ \hline
pi       & $\pi$        & $\Pi$        \\ \hline   \hline

rho      & $\rho$       & \blank       \\ \hline
sigma    & $\sigma$     & $\Sigma$     \\ \hline
tau      & $\tau$       & \blank       \\ \hline
upsilon  & $\upsilon$   & $\Upsilon$   \\ \hline   \hline

phi      & $\phi$       & $\Phi$       \\ \hline
chi      & $\chi$       & \blank       \\ \hline
psi      & $\psi$       & $\Psi$       \\ \hline
omega    & $\omega$     & $\Omega$     \\ \hline
\end{tabular}
\caption{A table depicting the greek letters available in math mode.}
\end{table}}


%~ \renewcommand{\vec}[1]{\boldsymbol{#1}}
%~ \newcommand{\vecN}[1]{\vec{\hat{#1}}}
\newcommand{\vecN}[1]{\hat{#1}}
\newcommand{\bs}[1]{\boldsymbol{#1}}
\newcommand{\Mu}[1]{\bs{#1}}
\newcommand{\CM}{\text{cm}}

\newcommand{\vvv}[3]{(#1,\,#2,\,#3)}
\newcommand{\ang}{\psi}
\newcommand{\vict}{\vec{w}}
\newcommand{\ax}{\vecN{x}}
\newcommand{\super}[2]{#1\textsuperscript{#2}}

\begin{document}
\title{Rotations and boost without matrices}
\author{Keith Pedersen}

\maketitle

{\it NOTE: This paper is a reprint from the Appendix of my doctoral thesis 
``Expanding the HEP frontier with boosted $b$-tags and the QCD power spectrum''.
It uses the nomenclature that $\vec{w}$ is a 3-vector, 
$\vecN{w}$ is a unit 3-vector, and $\Mu{p}$ is a 4-vector.}
\medskip

Given my investigation into unsafe operations in floating point arithmetic, 
I decided to write my own C++ vector classes, so that I could be assured that
they internally respect floating point limitations
(e.g., the options available with ROOT use {\tt acos} to find interior angle).
My {\tt Vector2}, {\tt Vector3}, and {\tt Vector4} classes 
store 2, 3, and 4-vectors (respectively), 
and are highly modified extensions of classes originally written by Z.~Sullivan.
In addition to defining basic functionality (add, scale, dot, cross, etc.),
I needed tools to rotate 3-vectors and boost 4-vectors;
this section presents the mathematical formalism I chose to use to 
accomplish those latter tasks.

{\it NOTE: The Rodrigues formula (both for rotations and boost) is shown for reference, 
whereas Section~\ref{sec:u-to-v} presents my own work.}

\section{Rotating vectors}%
%
The most natural method to rotate 3-vectors is via $3\times3$ square matrices.
However, it can be cumbersome to implement these matrices for an arbitrary rotation, 
since they require hard-coding large trigonometric expressions.
The size of these expressions lends itself to the possibility of
floating point cancellation,
and ensuring their numerical stability is a headache.
However, one does not have to use matrices to define rotations;
the Rodrigues formula rotates using only the dot and cross product.

\subsection{The Rodrigues rotation formula}%
%
To implement an active, right-handed (RH) rotation of vector $\vec{w}$
by some angle $\ang$ about some axis $\ax$,
the victim  is projected into two pieces:
its longitudinal length and transverse vector
\begin{align}
	w_L\ds & = \vict\cdot\ax\,,\label{eq:v_L}\\
	\vec{w}_T\ds & = \vict - w_L\ds \ax \label{eq:v_T}
	\,.
\end{align}
We can then calculate another transverse $\vec{w}_T\ds$, rotated by $90^\circ$ relative to the first;
\begin{equation}
	\vec{w}_{T^{\,\prime}}\ds = \ax\times\vict
	\,.
\end{equation}
The two transverse $\vec{w}$ form a basis for the rotation, 
which $\ax$ is unaltered by;
this allows us to calculate the rotated vector quite simply
\begin{equation}\label{eq:rot}
	%~ \vict^\prime=R(\vict) = \underset{\vec{w}_L\ds}{\underbrace{(\vict\cdot\ax)\ax}} 
		%~ + \cos\ang\underset{\vec{w}_T\ds}{\underbrace{(\vict - \vec{w}_L\ds)}}
		%~ + \sin\ang\underset{\vec{w}_{T^{\prime}}\ds}{\underbrace{(\ax\times\vict)}}
		%~ \,.
	\vict^\prime=R(\vict) = w_L\ds \ax + \cos\ang\, \vec{w}_T\ds + \sin\ang\,\vec{w}_{T^{\,\prime}}\ds
		\,.
\end{equation}

We can validate this scheme by showing it preserves %$\abs{\vict}^2$ and $\vict\cdot\ax$, 
the defining properties of rotations. 
First, the angle to the axis is unaltered;
\begin{align}
	\vict^\prime \cdot\ax
	& = (\vict\cdot\ax)\cancelto{1}{\ax\cdot\ax}
	  + \cos\ang(\vict\cdot\ax - (\vict\cdot\ax)\cancelto{1}{\ax\cdot\ax})	
	  + \sin\ang\cancelto{0}{((\ax\times\vict)\cdot\ax)}		\nonumber\\
	& = \vict\cdot\ax
	\,.
\end{align}
(since $\vec{a}\cdot(\vec{b}\times\vec{c}) 
	= \vec{b}\cdot(\vec{c}\times\vec{a}) 
	= \vec{c}\cdot(\vec{a}\times\vec{b})$).
Second, the vector's length is unaltered;
\begin{align}
	\abs{\vict^\prime}^2 
	%~ & = (\vict\cdot\ax)^2\cancelto{1}{\ax\cdot\ax}
	 %~ + \cos^2\ang\,(\abs{\vict}^2 - 2(\vict\cdot\ax)^2 + (\vict\cdot\ax)^2)
	  %~ + \sin^2\ang\,(\abs{\vict}^2 - (\vict\cdot\vecN{x})^2)	\nonumber\\
	  & = (\vict\cdot\ax) + (\sin^2\ang + \cos^2\ang)(\abs{\vict}^2 - (\vict\cdot\ax)^2)	\nonumber\\
	  & = \abs{\vict}^2
	  \,.
\end{align}
This result uses the identity 
$\abs{\vec{u}\times\vec{v}}^2=\abs{\vec{u}}^2\abs{\vec{v}}^2 - (\vec{u}\cdot\vec{v})^2$ and
%~ (from ${\epsilon_{ijk}\ds\epsilon_{ilm}\ds=\delta_{jl}\ds\delta_{km}\ds-\delta_{jm}\ds\delta_{kl}\ds}$), 
implicitly draws upon the mutual orthogonality of
$\vec{w}_L\ds$, $\vec{w}_T\ds$ and $\vec{w}_{T^{\,\prime}}\ds$.

%~ \begin{align}
	%~ \vec{w}_\parallel\ds \cdot \vec{w}_\bot\ds
		%~ & =(\vict\cdot\ax)^2 - (\vict\cdot\ax)^2 = 0
	%~ \,,\\
	%~ \vec{w}_\bot\ds \cdot \vec{w}^\prime_\bot
		%~ & = \cancelto{0}{\vec{w}\cdot(\ax\times\vec{w})} 
			%~ - (\vict\cdot\ax)\,\cancelto{0}{\ax\cdot(\ax\times\vec{w})} = 0
	%~ \,.
%~ \end{align}

\subsection{The hidden degree for freedom when rotating $\bs{\vec{u}}$ to $\bs{\vec{v}}$}%
\label{sec:u-to-v}%
%
Instead of an axis $\ax$ and angle $\ang$ as our input degrees of freedom,
we may want the rotation that takes vector $\vec{u}\to\vec{v}$.
We can reuse Equation~\ref{eq:rot}, and define:
\begin{align}
	%~ \cos\ang & = \frac{\vec{u}\cdot\vec{v}}{\abs{\vec{u}}\abs{\vec{v}}}\\
	%~ \sin\ang & = \frac{\abs{\vec{u}\times\vec{v}}}{\abs{\vec{u}}\abs{\vec{v}}}\\
	%~ \ax_1\ds & = \frac{\vec{u}\times\vec{v}}{\abs{\vec{u}\times\vec{v}}}\label{eq:x1}
	\cos\ang & = \vecN{u}\cdot\vecN{v}
	\,,\\
	\sin\ang & = \abs{\vec{u}\times\vec{v}}
	\,,\\
	\ax_1\ds & = \frac{\vecN{u}\times\vecN{v}}{\abs{\vecN{u}\times\vecN{v}}}
		= \frac{\vecN{u}\times\vecN{v}}{\sin\ang}\label{eq:x1}
	\,.
\end{align}
However, $\ax_1\ds$ is only one \emph{possible} rotation axis which takes 
$\vec{u}\to \vec{v}$. Consider a rotation of $\ang = \pi$ 
about the axis bisecting the two normalized vectors
\begin{equation}\label{eq:x2}
	\ax_2\ds = \frac{\vecN{u} + \vecN{v}}{\abs{\vecN{u} + \vecN{v}}} 
		= \frac{\vecN{u} + \vecN{v}}{\sqrt{2(1+\vecN{u}\cdot\vecN{v})}}
	\,.
\end{equation}
This rotation also clearly takes $\vec{u}\to\vec{v}$.
In fact, any axis which satisfies 
\begin{equation}
	\vecN{u}\cdot\ax = \vecN{v}\cdot\ax
\end{equation}
defines a valid rotation (since $\vec{u}$ and $\vec{v}$ will have 
the same latitude relative to $\ax$, 
and thus trace out the same circle during the rotation).
The ``rotation which takes $\vec{u}\to\vec{v}\,$'' is ambiguous!

This ambiguity stems from a hidden degree of freedom.
We are free to choose a coordinate system where
\begin{align}
	\vec{u} & = \vvv{0}{0}{1}
	\,,\\
	\vec{v} & = \vvv{\cos \phi \sin \theta}{\sin \phi \sin \theta}{\cos\theta}
	\,.
\end{align}
We can get from $\vec{u}$ to $\vec{v}$ in two steps:
(i)~a RH rotation about $\vecN{z}$ by angle $\phi$ and
(ii)~a RH rotation about $\vecN{y}^\prime$ (the new $y$-axis, after the first rotation) by angle $\theta$.
This procedure uses only two of the three Euler angles required to 
cover SO(3) (the group of 3-dimensional rotations).
To complete the coverage, we need a final RH rotation about $\vecN{z}^{\prime\prime}$ 
(the final $z$-axis, which in our case is the newly minted $\vec{v}$).

By construction, this post-rotation about $\vec{v}$ by angle $\omega$ 
cannot alter $\vec{v}$, so it does not spoil
the original purpose of this rotation (take $\vec{u}\to\vec{v}$).
Instead, $\omega$ determines what happens to \emph{every other} vector, 
and does so by selecting \emph{one} axis $\ax$ from the set which map $\vec{u}\to\vec{v}$.
If we can find this~$\ax$, we can describe the complete operation as a single rotation 
(instead of two sequential rotations).
We have already found two valid $\ax$ (Eqs.~\ref{eq:x1} and \ref{eq:x2}), 
and they are fortuitously orthogonal, so we can use them to construct
a basis that parameterizes all possible axes of rotation
\begin{equation}
	\ax = a\,\ax_1\ds + b\,\ax_2\ds
	\,.
	%~ \qquad\qquad(\text{where }a^2+b^2=1)\,.
\end{equation}
Our task is now clear; given $\vec{u}$, $\vec{v}$ and $\omega$, 
determine $a$ and $b$ to find $\ax$.

$R_1\ds$ is the rotation about $\ax_1\ds$
by $\theta=\arccos(\vecN{u}\cdot\vecN{v})$ and $R_2\ds$ is the rotation about $\ax_2\ds$ by $\omega$.
Applying them consecutively produces a composite rotation
which cannot alter the axis of rotation;
\begin{equation}
	\ax = R_2\ds(R_1\ds(\ax))\,.
\end{equation}
Since this defining property applies to both of $\ax$'s component individually (and linearly), 
we can solve for $a$ and $b$ by using unitarity as one equation (i.e., ${a^2+b^2=1}$), 
then obtain the other equation by calculating
\begin{align}\label{eq:a}
	a & = R_2\ds(R_1\ds(a\,\ax_1\ds + b\,\ax_2\ds))\cdot\ax_1\ds\nonumber\\
		& = a\,R_2\ds(R_1\ds(\ax_1\ds))\cdot\ax_1\ds + b\,R_2\ds(R_1\ds(\ax_2\ds))\cdot\ax_1\ds\,.
	%~ b\,\ax_2\ds & = R_2\ds(R_1\ds(b\,\ax_2\ds)) = b\,R_2\ds(R_1\ds(\ax_2\ds))\,,
\end{align}
%~ In doing so, we only need to keep terms parallel to $\ax_1\ds$.

Beginning with $\ax_1\ds$, we are lucky that 
$R_1\ds$ does not alter its own axis, 
while $R_2\ds$ creates only one term parallel to $\ax_1\ds$;
\begin{align}\label{eq:part-x1}
	R_1\ds(\ax_1\ds) & = \ax_1\ds\,
	\,;\\
	R_2\ds(R_1\ds(\ax_1\ds))\cdot\ax_1\ds
	%~ = R_2\ds(\ax_1\ds)\cdot\ax_1\ds
	& = (\cancelto{0}{(\ax_1\cdot\vecN{v})}\,\vecN{v}
		+ \cos\omega(\ax_1\ds - 0) 
		+ \sin\omega\underset{\bot\text{ to }\ax_1\ds}{\underbrace
			{(\vecN{v}\times\ax_1\ds)}})\cdot\ax_1\ds = \cos\omega
	\,.
\end{align}
The effect on $\ax_2\ds$ is slightly more complicated;
\begin{align}\label{eq:R1-x2}
	R_1\ds(\ax_2\ds) 
		& = (\cancelto{0}{(\ax_2\ds\cdot\ax_1)}\,\ax_1
		+ \cos\theta(\ax_2\ds - 0) + \sin\theta(\ax_1\ds\times\ax_2\ds)\nonumber\\
		& = \frac{1}{\sqrt{2(1+\cos(\theta))}}
		\left(\cos\theta\,(\vecN{u}+\vecN{v}) 
			+ \sin\theta\,\frac{(\vecN{u}\times\vecN{v})}{\sin\theta}\times(\vecN{u}+\vecN{v})\right)\nonumber\\
		& = \frac{1}{\sqrt{2(1+\cos(\theta))}}
		\left(\cos\theta\,(\vecN{u}+\vecN{v}) + 
		\vecN{v} - \vecN{u}\cos\theta + \vecN{v}\cos\theta - \vecN{u}\right)\nonumber\\
		& = \frac{1}{\sqrt{2(1+\cos(\theta))}}
		\left((2\cos\theta + 1)\vecN{v} - \vecN{u}\right)
	\,,
\end{align}
using ${(\vec{a}\times\vec{b}\,)\times\vec{c} 
	= \vec{b}\,(\vec{a}\cdot\vec{c}\,) - \vec{a}\,(\vec{b}\cdot\vec{c}\,)}$.
Before we put all of Equation~\ref{eq:R1-x2} through $R_2\ds$, 
we should recall that the final equation uses only terms parallel to $\ax_1\ds$. 
$R_2\ds$ returns only terms which are 
(i)~parallel to the incoming vector 
(which in this case is in the $uv$-plane, and thus orthogonal to~$\ax_1\ds$), 
(ii)~parallel to $\vecN{v}$ (also in the $uv$ plane) and 
(iii)~perpendicular to $\vecN{v}$ (via $\vecN{v}\times\vict$).
Hence, only the \super{3}{rd} piece is meaningful, 
and since $\vecN{v}\times\vecN{v}=0$, 
we only need to give it Equation~\ref{eq:R1-x2}'s $\vecN{u}$ term;
\begin{align}\label{eq:part-x2}
	R_2\ds(R_1\ds(\ax_2\ds))\cdot\ax_1\ds
		& = \frac{1}{\sqrt{2(1+\cos(\theta))}} \,R_2\ds(-\vecN{u})\cdot\ax_1\ds
		= \frac{1}{\sqrt{2(1+\cos(\theta))}} (\vecN{v}\times(-\vecN{u}))\cdot\ax_1\ds\nonumber\\
		& = \sin\omega\frac{\sin\theta}{\sqrt{2(1+\cos(\theta))}}
	\,,
\end{align}
using $\vecN{u}\times\vecN{v} = \sin\theta\,\ax_1\ds$.
Combining Equation~\ref{eq:a},~\ref{eq:part-x1} and~\ref{eq:part-x2} 
we get our system of equations
\begin{align}
	a & = a \cos\omega + b\sin\omega\frac{\sin\theta}{\sqrt{2(1+\cos(\theta))}}
	\,,\\
	1 &= a^2 + b^2
	\,.
\end{align}

Having done the hard work (see the Appendix),
we can plug our system of equations into Mathematica to obtain
\begin{align}
	a & = 2\cos(\omega/2)\frac{\sin(\theta/2)}{c(\theta)}
	\,,\\
	b & = 2\sin(\omega/2)\frac{1}{c(\theta)}
	\,,\\
	c(\theta) & = \sqrt{3 - \cos(\omega) - \cos(\theta)(1+\cos(\omega))}
	\,.
\end{align}
I then used Mathematica to validate this solution by checking that 
$\ax$ matches the eigenvector of the composite rotation $R_2\ds(R_1\ds(\vict))$
(up to the parity operation ${\ax \to -\ax}$,
an irreducibly ambiguity of the eigenvector).

However, $c(\theta)$ is numerically unstable if used na\"ively, 
due to the cosine cancellations. These terms should be rewritten 
in a form which \emph{doubles} the precision of the floating point result
\begin{equation}\label{eq:1-cos}
	1-\cos(t) \mapsto 2\sin^2(t/2)\,.
\end{equation}
This gives us
\begin{equation}
	c(\theta)\mapsto
	\sqrt{2\sin^2(\theta/2) + 2\sin^2(\omega/2) + (1 - \cos(\theta)\cos(\omega))}\,.
\end{equation}
The final cancellation can be corrected using
\begin{align}
	\cos(\theta)\cos(\omega) 
		& = \frac{1}{2}(\cos(\theta+\omega) + \cos(\theta-\omega))\,;\\
	(1 - \cos(\theta)\cos(\omega)) 
		&\mapsto \frac{1}{2}(1-\cos(\theta+\omega))
		 + \frac{1}{2}(1-\cos(\theta-\omega))\,.
\end{align}
Again using Equation~\ref{eq:1-cos}, we obtain the final expression
\begin{equation}
	c(\theta)\mapsto
	\sqrt{2\sin^2(\theta/2) + 2\sin^2(\omega/2)	
	+ \sin^2(\theta+\omega) 
	+ \sin^2(\theta-\omega)}\,.
\end{equation}

Given $\vec{u}$, $\vec{v}$ and $\omega$, we now have the tools to find the axis $\ax$
about which the composite rotations occur.\footnote
{\nobreak
	There is one class of system where $\ax$ remains ambiguous;
	when $\vec{u}$ and $\vec{v}$ are antiparallel, 
	$\ax_1\ds$ and $\ax_2\ds$ are both null.
	If $\vec{x}$ is supplied externally, 
	$\omega$ rotates it around the shared $uv$ axis,
	but $\vec{x}$ cannot be determined from $\omega$ alone.
}
But what is the angle $\ang$ of rotation?
We can determine $\psi$ empirically by projecting 
$\vecN{u}$ and $\vecN{v}$ into the plane of rotation (e.g. 
$\vecN{u}_\bot\ds = \vecN{u} - (\vecN{u}\cdot\vecN{x}) \vecN{x}$).
The rotation angle is then defined via
\begin{equation}
	\psi = \text{atan2}(\sin\psi, \cos\psi)
		= \text{atan2}\left(\text{sign}(a)\abs{\vecN{u}_\bot\ds \times \vecN{v}_\bot\ds},\; 
		\vecN{u}_\bot\ds \cdot \vecN{v}_\bot\ds\right)\,.
\end{equation}
It is best to use $\text{atan2}$ because it is more precise
for angles near $0$, $\pi/2$ and $\pi$. Note that we have to 
inject the \emph{sign} of $a$ into $\sin(\psi)$, 
because when $a<0$, the RH rotation becomes larger than 
$\pi$, so we must instead uses a \emph{negative} RH rotation.

%~ An effective implementation should keep in mind that 
%~ many of these calculations are redundant (or terms cancel).
%~ These operations are numerically stable, provided that $\vec{u}$ is 
%~ not exactly parallel/anti-parallel, a condition that can be checked 
%~ before constructing $\vec{x}$.

I have tested an implementation of this algorithm and it works quite well
(it is both length and angle preserving). 
I have additionally validated that rotating once about $\ax$ gives 
the same result as the two-step, composite rotation.
%~ about $\vecN{x}_1\ds$ and $\vecN{x}_2\ds$.
%~ The main problem is a slight change in magnitude when $\vec{u}$ and $\vec{v}$
%~ are mostly anti-parallel. This can be fixed with a
%~ magnitude preserving step that adds a 15\% overhead.

\section{Boosting between frames}%
%
Since SO(3) is a sub-group of the Lorentz group, 
it is not surprising that there is a boost analog of the Rodrigues formula.
A four momentum $\Mu{p}$ (with energy $p^0$ and 3-momentum $\vec{p}$\,)
can be boosted by some speed~$\vec{\beta}$ by 
splitting $\vec{p}$ into its longitudinal~$p_L\ds$ and
transverse momentum~$\vec{p}_T\ds$ relative to $\vecN{\beta}$ 
(see Eqs.~\ref{eq:v_L}~and~\ref{eq:v_T}).
%~ \begin{align}
	%~ p_L\ds & = \vec{p}\cdot\vecN{\beta}\\
	%~ \vec{p}_T\ds & = \vec{p} - p_L\ds\vecN{\beta}
%~ \end{align}
Unlike the Rodrigues rotation formula, we do not need to find 
a second perpendicular axis, because the ``rotation'' is 
always between the longitudinal momentum and energy/time.
This allows us to use the standard definition of the Lorentz boost matrix, 
but with a coordinate system defined by $\vecN{\beta}$;
writing $p^0$ and $p_L\ds$ as scalars,
but keeping the transverse momentum $\vec{p}_T\ds$ as a vector
\begin{align}
	\Mu{p}^\prime = 
	\begin{pmatrix}
		\gamma & 0 & \beta\gamma \\
		%~ 0 & \left(\begin{smallmatrix}1& 0 \\ 0 & 1\\\end{smallmatrix}\right) & 0 \\
		%~ 0 & 1 & 0 \\
		0 & \ident & 0 \\
		\beta\gamma & 0 & \gamma \\
	\end{pmatrix}
	\begin{pmatrix}
		p^0 \\
		\vec{p}_T\ds \\
		p_L\ds \\
	\end{pmatrix}
	& = 
	\begin{pmatrix}
		\gamma p^0 + \beta\gamma \,p_L\ds \\
		\vec{p}_T\ds \\
		\beta\gamma \,p^0 + \gamma p_L\ds \\
	\end{pmatrix}\\
	& =
	\begin{pmatrix}
		p^0 \\
		\vec{p}_T\ds \\
		p_L\ds \\
	\end{pmatrix}
	+
	\begin{pmatrix}
		(\gamma - 1)p^0 + \beta\gamma \,p_L\ds \\
		\vec{0} \\
		\beta\gamma \,p^0 + (\gamma - 1)p_L\ds \\
	\end{pmatrix}\,.
\end{align}
Hence, adding the correct four-momenta implements the desired boost;
\begin{align}
	\Mu{p}^\prime & = \Mu{p} + \Mu{b}
	\,,\\
	b^0 &= (\gamma - 1)p^0 + \beta\gamma \,p_L\ds
	\,,\\
	\vec{b} &= 	\left(\beta\gamma \,p^0 + (\gamma - 1)p_L\ds\right)\vecN{\beta}
	\,.
\end{align}

There are intrinsic floating point limitations to this boost scheme.
When an ultra-relativistic particle $\Mu{p}$ is boosted back to its CM frame,
machine $\epsilon$ relative rounding errors when adding $\Mu{b}$ to $\Mu{p}$
can cause relative errors of $\bigO{\gamma\,\epsilon}$ in $p^0_\CM$
(and absolute errors at the same order in $\vec{p}_\CM\ds$, 
which should always be a null vector in the CM frame).

\appendix*
\section{The apology}

{\nobreak
	I see three steps in answering any question scientifically:

	\begin{enumerate}
		\item Write down a model which describes the system with sufficient accuracy.\label{model}
		
		\item Try and fail ad naseum until the Eureka moment
		allows you to \emph{describe} the solution.\label{sol-des}
		
		\item Find the solution.\label{sol}
	\end{enumerate}
	%
	The difference between step~\ref{sol-des}~and~\ref{sol} is subtle, but important.
	A \emph{description} of the solution is a complete statement of 
	\emph{where} the solution can be found, its GPS coordinates ---
	it is a differential equation, 
	an unevaluated integral, the polynomial whose roots we will find.
	Step~\ref{sol} is the vehicle that takes us there ---
	an algorithm, a clever change of variable, an analytic continuation. 
	It is math, and frequently the heavy kind.
	
	In my opinion, competence and creativity in step~\ref{model}~and~\ref{sol-des}
	are the mark of a good scientist. 
	This requires a vast array of knowledge and experience
	(e.g.\ you have to know what an eigenvalue is,
	and have previously used them to solve simple problems, 
	before you can use them to describe the solution for some novel problem).
	But once we reach step~\ref{sol} (calculate the eigenvalue),
	we should beg, borrow and steal. Of course, someone has to 
	blaze the original trail when a new technique is found, 
	and for that they earn their reverence in the pages of a textbook.
	But since they bled to build that trail, 
	we should have the common decency to use it.
	Conversely, each of us should forge our own trail for the
	step~\ref{sol} in which \emph{we} are the experts, and then tell the world.
	
	But we cannot be experts in the huge library of step~\ref{sol}'s.
	Furthermore, since step~\ref{sol}'s are often quite general, 
	they lend themselves to automation.
	Computers are still not very good at answering \emph{declarative} statements like
	``what is the volume of a sphere'' (and here I don't mean using a 
	search engine or a neural net to find a solution published by a human).
	However, Mathematica will swiftly provide the 
	symbolic solution to a very \emph{imperative} statement
	\begin{equation*}
		V_\text{sphere} = \int_0^R\diff{r}\int_0^\pi\diff{\theta}\int_0^{2\pi}\diff{\phi}
		 \,r^2\sin(\theta) = \frac{4}{3}\pi\,R^3\,.
	\end{equation*}
	So when I find myself with a system of equations which I can 
	ask the computer to solve (nearly instantaneously),
	I use the computer --- that's what computers are for.
}

\end{document}
