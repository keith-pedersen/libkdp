\documentclass{article}
\usepackage{amsmath}
\usepackage[margin=3.4cm]{geometry}
\usepackage{graphicx}
\usepackage{mathdots} % for \iddots
\pagenumbering{gobble}
\usepackage{revsymb} % openone
\usepackage{cancel}

%~ \newcommand{\vecN}[1]{\hat{#1}}
\renewcommand{\vec}[1]{\boldsymbol{#1}}
\newcommand{\vecN}[1]{\vec{\hat{#1}}}

\newcommand{\ds}{^{}}
\newcommand{\bigO}[1]{\mathcal{O}(#1)}
\newcommand{\abs}[1]{|#1|}
\newcommand{\vvv}[3]{(#1,\,#2,\,#3)}
\newcommand{\ang}{\psi}
\newcommand{\vict}{\vec{w}}
\newcommand{\ax}{\vecN{x}}
\newcommand{\super}[2]{#1\textsuperscript{#2}}
%~ \newcommand{\openone}{\scalebox{0.91}{1}\!\!1}

\begin{document}
\title{An arbitrary rotation without matrices}
\author{Keith Pedersen}

\maketitle

{\it NOTE: This paper uses the nomenclature that $\vec{w}$ is a vector and 
$\vecN{w}$ is a unit vector.}
\medskip

A right-handed (RH) rotation, by some angle $\ang$ about some axis $\ax$, 
can be defined without matrices. The victim vector $\vec{w}$ is projected into three pieces:
($\vec{w}_\parallel\ds$) the piece parallel to the axis, 
($\vec{w}_{\bot}\ds$) the piece perpendicular to the axis, 
and ($\vec{w}^\prime_{\bot}$), the perpendicular $\vec{w}_{\bot}\ds$ rotated by $90^\circ$.
The parallel piece is unaltered by the rotation, 
and the two $\vec{w}_{\bot}\ds$ form a basis for the rotation;
\begin{equation}\label{eq:rot}
	\vict^\prime=R(\vict) = \underset{\vec{w}_\parallel\ds}{\underbrace{(\vict\cdot\ax)\ax}} 
		+ \cos\ang\underset{\vec{w}_{\bot}\ds}{\underbrace{(\vict - \vec{w}_\parallel\ds)}}
		+ \sin\ang\underset{\vec{w}^\prime_{\bot}}{\underbrace{(\ax\times\vict)}}
		\,.
\end{equation}
We can validate this scheme by showing it preserves %$\abs{\vict}^2$ and $\vict\cdot\ax$, 
the defining properties of rotations. 
First, the angle to the axis is unaltered;
\begin{align}
	\vict^\prime \cdot\ax
	& = (\vict\cdot\ax)\cancelto{1}{\ax\cdot\ax}
	  + \cos\ang(\vict\cdot\ax - (\vict\cdot\ax)\cancelto{1}{\ax\cdot\ax})	
	  + \sin\ang\cancelto{0}{((\ax\times\vict)\cdot\ax)}		\nonumber\\
	& = \vict\cdot\ax
\end{align}
(which uses $\vec{a}\cdot(\vec{b}\times\vec{c}) 
	= \vec{b}\cdot(\vec{c}\times\vec{a}) 
	= \vec{c}\cdot(\vec{a}\times\vec{b})$).
Second, the vector's length is unaltered;
\begin{align}
	\abs{\vict^\prime}^2 
	& = (\vict\cdot\ax)^2\cancelto{1}{\ax\cdot\ax}
	 + \cos^2\ang\,(\abs{\vict}^2 - 2(\vict\cdot\ax)^2 + (\vict\cdot\ax)^2)
	  + \sin^2\ang\,\abs{\vict}^2(\abs{\vict}^2 - (\vict\cdot\vecN{x})^2)	\nonumber\\
	  & = (\vict\cdot\ax) + (\sin^2\ang + \cos^2\ang)(\abs{\vict}^2 - (\vict\cdot\ax)^2)	\nonumber\\
	  & = \abs{\vict}^2
\end{align}
Here, we have used the identity 
$\abs{\vec{u}\times\vec{v}}^2=\abs{\vec{u}}^2\abs{\vec{v}}^2 - (\vec{u}\cdot\vec{v})^2$
(from $\epsilon_{ijk}\ds\epsilon_{ilm}\ds=\delta_{jl}\ds\delta_{km}\ds-\delta_{jm}\ds\delta_{kl}\ds$), 
and implicitly used the orthogonality of the each component vector
\begin{align}
	\vec{w}_\parallel\ds \cdot \vec{w}_\bot\ds
		& =(\vict\cdot\ax)^2 - (\vict\cdot\ax)^2 = 0
	\,,\\
	\vec{w}_\bot\ds \cdot \vec{w}^\prime_\bot
		& = \cancelto{0}{\vec{w}\cdot(\ax\times\vec{w})} 
			- (\vict\cdot\ax)\,\cancelto{0}{\ax\cdot(\ax\times\vec{w})} = 0
	\,.
\end{align}

Instead of an axis $\ax$ and angle $\ang$ as our input degrees of freedom,
we may want the rotation that takes vector $\vec{u}\to\vec{v}$.
We can reuse Eq.~\ref{eq:rot}, and define:
\begin{align}
	%~ \cos\ang & = \frac{\vec{u}\cdot\vec{v}}{\abs{\vec{u}}\abs{\vec{v}}}\\
	%~ \sin\ang & = \frac{\abs{\vec{u}\times\vec{v}}}{\abs{\vec{u}}\abs{\vec{v}}}\\
	%~ \ax_1\ds & = \frac{\vec{u}\times\vec{v}}{\abs{\vec{u}\times\vec{v}}}\label{eq:x1}
	\cos\ang & = \vecN{u}\cdot\vecN{v}\\
	\sin\ang & = \abs{\vec{u}\times\vec{v}}\\
	\ax_1\ds & = \frac{\vecN{u}\times\vecN{v}}{\abs{\vecN{u}\times\vecN{v}}}
		= \frac{\vecN{u}\times\vecN{v}}{\sin\ang}\label{eq:x1}
\end{align}
However, $\ax_1\ds$ is only one \emph{possible} rotation axis which takes 
$\vec{u}\to \vec{v}$. Consider a rotation of $\ang = \pi$ 
about the axis bisecting the two normalized vectors
\begin{equation}\label{eq:x2}
	\ax_2\ds = \frac{\vecN{u} + \vecN{v}}{\abs{\vecN{u} + \vecN{v}}} 
		= \frac{\vecN{u} + \vecN{v}}{\sqrt{2(1+\vecN{u}\cdot\vecN{v})}}
	\,.
\end{equation}
This rotation also clearly takes $\vec{u}\to\vec{v}$.
In fact, any axis which satisfies 
\begin{equation}
	\vecN{u}\cdot\ax = \vecN{v}\cdot\ax
\end{equation}
defines a valid rotation (since $\vec{u}$ and $\vec{v}$ will have 
the same latitude relative to $\ax$, 
and thus trace out the same circle during the rotation).
Thus, ``the rotation which takes $\vec{u}\to\vec{v}\,$'' is ambiguous.

This ambiguity stems from a hidden degree of freedom.
We are free to choose $\vvv{x}{y}{z}$ vectors 
\begin{align}
	\vec{u} & = \vvv{0}{0}{1}\\
	\vec{v} & = \vvv{\cos \phi \sin \theta}{\sin \phi \sin \theta}{\cos\theta}
\end{align}
We can get from $\vec{u}$ to $\vec{v}$ in two steps:
(i) A RH rotation about $\vecN{z}$ by angle $\phi$.
(ii) A RH rotation about $\vecN{y}^\prime$ (the new $y$-axis, after the first rotation) by angle $\theta$.
This procedure uses only two of the three Euler angles required to 
cover SO(3), the group of 3-dimensional rotations.
To complete the coverage, we need a final RH rotation about $\vecN{z}^{\prime\prime}$ 
(the final $z$-axis, which in our case is the newly minted $\vec{v}$).

By construction, this post-rotation about $\vec{v}$ by angle $\omega$ 
cannot alter $\vec{v}$, so it does not spoil
the original purpose of this rotation (take $\vec{u}\to\vec{v}$).
Instead, $\omega$ determines what happens to \emph{every other} vector, 
and does so by selecting \emph{one} axis $\ax$ from the set which map $\vec{u}\to\vec{v}$.
If we can find this~$\ax$, we can describe the complete operation as a single rotation 
(instead of two sequential rotations).
We have already found two valid $\ax$ (Eqs.~\ref{eq:x1} and \ref{eq:x2}), 
and they are fortuitously orthogonal, so we can use them to construct
a basis that parameterizes all possible axes of rotation
\begin{equation}
	\ax = a\,\ax_1\ds + b\,\ax_2\ds
	%~ \qquad\qquad(\text{where }a^2+b^2=1)\,.
\end{equation}
Our task is now clear; given $\vec{u}$, $\vec{v}$ and $\omega$, 
determine $a$ and $b$ to find $\ax$.

$R_1\ds$ is the rotation about $\ax_1\ds$
by $\theta=\arccos(\vecN{u}\cdot\vecN{v})$ and $R_2\ds$ is the rotation about $\ax_2\ds$ by $\omega$.
Applying them consecutively produces a composite rotation
which cannot alter the axis of rotation;
\begin{equation}
	\ax = R_2\ds(R_1\ds(\ax))\,.
\end{equation}
Since this defining property applies to both of $\ax$'s component individually (and linearly), 
we can solve for $a$ and $b$ by using unitarity as one equation (i.e.\ ${a^2+b^2=1}$), 
then obtain the other equation by calculating
\begin{align}\label{eq:a}
	a & = R_2\ds(R_1\ds(a\,\ax_1\ds + b\,\ax_2\ds))\cdot\ax_1\ds\nonumber\\
		& = a\,R_2\ds(R_1\ds(\ax_1\ds))\cdot\ax_1\ds + b\,R_2\ds(R_1\ds(\ax_2\ds))\cdot\ax_1\ds\,.
	%~ b\,\ax_2\ds & = R_2\ds(R_1\ds(b\,\ax_2\ds)) = b\,R_2\ds(R_1\ds(\ax_2\ds))\,,
\end{align}
%~ In doing so, we only need to keep terms parallel to $\ax_1\ds$.

Beginning with $\ax_1\ds$, we are lucky that 
$R_1\ds$ does not alter it's own axis, 
while $R_2\ds$ creates only one term parallel to $\ax_1\ds$;
\begin{align}\label{eq:part-x1}
	R_1\ds(\ax_1\ds) & = \ax_1\ds\,\,;\\
	R_2\ds(R_1\ds(\ax_1\ds))\cdot\ax_1\ds
	%~ = R_2\ds(\ax_1\ds)\cdot\ax_1\ds
	& = (\cancelto{0}{(\ax_1\cdot\vecN{v})}\,\vecN{v}
		+ \cos\omega(\ax_1\ds - 0) 
		+ \sin\omega\underset{\bot\text{ to }\ax_1\ds}{\underbrace
			{(\vecN{v}\times\ax_1\ds)}})\cdot\ax_1\ds = \cos\omega
\end{align}
The effect on $\ax_2\ds$ is slightly more complicated;
\begin{align}\label{eq:R1-x2}
	R_1\ds(\ax_2\ds) 
		& = (\cancelto{0}{(\ax_2\ds\cdot\ax_1)}\,\ax_1
		+ \cos\theta(\ax_2\ds - 0) + \sin\theta(\ax_1\ds\times\ax_2\ds)\nonumber\\
		& = \frac{1}{\sqrt{2(1+\cos(\theta))}}
		\left(\cos\theta\,(\vecN{u}+\vecN{v}) 
			+ \sin\theta\,\frac{(\vecN{u}\times\vecN{v})}{\sin\theta}\times(\vecN{u}+\vecN{v})\right)\nonumber\\
		& = \frac{1}{\sqrt{2(1+\cos(\theta))}}
		\left(\cos\theta\,(\vecN{u}+\vecN{v}) + 
		\vecN{v} - \vecN{u}\cos\theta + \vecN{v}\cos\theta - \vecN{u}\right)\nonumber\\
		& = \frac{1}{\sqrt{2(1+\cos(\theta))}}
		\left((2\cos\theta + 1)\vecN{v} - \vecN{u}\right)
\end{align}
(using ${(\vec{a}\times\vec{b}\,)\times\vec{c} 
	= \vec{b}\,(\vec{a}\cdot\vec{c}\,) - \vec{a}\,(\vec{b}\cdot\vec{c}\,)}$).
Before we put all of Eq.~\ref{eq:R1-x2} through $R_2\ds$, 
we should recall that the final equation only uses terms parallel to $\ax_1\ds$. 
$R_2\ds$ only returns terms which are 
(i)~parallel to the incoming vector 
(which in this case is in the $uv$-plane, and thus orthogonal to~$\ax_1\ds$), 
(ii)~parallel to $\vecN{v}$ (also in the $uv$ plane) and 
(iii)~perpendicular to $\vecN{v}$ (via $\vecN{v}\times\vict$).
Hence, only the \super{3}{rd} piece is meaningful, 
and since $\vecN{v}\times\vecN{v}=0$, 
we only need to give it Eq.~\ref{eq:R1-x2}'s $\vecN{u}$ term;
\begin{align}\label{eq:part-x2}
	R_2\ds(R_1\ds(\ax_2\ds))\cdot\ax_1\ds
		& = \frac{1}{\sqrt{2(1+\cos(\theta))}} \,R_2\ds(-\vecN{u})\cdot\ax_1\ds
		= \frac{1}{\sqrt{2(1+\cos(\theta))}} (\vecN{v}\times(-\vecN{u}))\cdot\ax_1\ds\nonumber\\
		& = \sin\omega\frac{\sin\theta}{\sqrt{2(1+\cos(\theta))}}
\end{align}
(using $\vecN{u}\times\vecN{v} = \sin\theta\,\ax_1\ds$).
Combining Eq.~\ref{eq:a},~\ref{eq:part-x1} and~\ref{eq:part-x2} we get
\begin{equation}
	a = a \cos\omega + b\sin\omega\frac{\sin\theta}{\sqrt{2(1+\cos(\theta))}}
\end{equation}

Having done the hard work, we can plug our system of equations into Mathematica to obtain
\begin{align}
	a & = 2\cos(\omega/2)\frac{\sin(\theta/2)}{c(\theta)}\\
	b & = 2\sin(\omega/2)\frac{1}{c(\theta)}\\
	c(\theta) & = \sqrt{3 - \cos(\omega) - \cos(\theta)(1+\cos(\omega))}
\end{align}
I then used Mathematica to validate this solution by checking that 
$\ax$ matches the eigenvector of the composite rotation $R_2\ds(R_1\ds(\vict))$.
However, $c(\theta)$ is numerically unstable if used na\"ively, 
due to the cosine cancellations. These terms should be rewritten 
in a form which \emph{doubles} the precision of the floating point result
\begin{equation}\label{eq:1-cos}
	1-\cos(t) \mapsto 2\sin^2(t/2)\,.
\end{equation}
This gives us
\begin{equation}
	c(\theta)\mapsto
	\sqrt{2\sin^2(\theta/2) + 2\sin^2(\omega/2) + (1 - \cos(\theta)\cos(\omega))}\,.
\end{equation}
The final cancellation can be corrected using
\begin{align}
	\cos(\theta)\cos(\omega) 
		& = \frac{1}{2}(\cos(\theta+\omega) + \cos(\theta-\omega))\,;\\
	(1 - \cos(\theta)\cos(\omega)) 
		&\mapsto \frac{1}{2}(1-\cos(\theta+\omega))
		 + \frac{1}{2}(1-\cos(\theta-\omega))\,.
\end{align}
Again using Eq.~\ref{eq:1-cos}, we obtain the final expression
\begin{equation}
	c(\theta)\mapsto
	\sqrt{2\sin^2(\theta/2) + 2\sin^2(\omega/2)	
	+ \sin^2(\theta+\omega) 
	+ \sin^2(\theta-\omega)}\,.
\end{equation}

Given $\vec{u}$, $\vec{v}$ and $\omega$, we now have the tools to find the axis $\ax$
about which the composite rotations occur.\footnote
{\nobreak
	There is one class of system where $\ax$ remains ambiguous;
	when $\vec{u}$ and $\vec{v}$ are antiparallel, 
	$\ax_1\ds$ and $\ax_2\ds$ are both null.
	If $\vec{x}$ is supplied externally, 
	$\omega$ rotates it around the shared $uv$ axis,
	but $\vec{x}$ cannot be determined from $\omega$ alone.
}
But what is the angle $\ang$ of rotation?
We can determine $\psi$ empirically by projecting 
$\vecN{u}$ and $\vecN{v}$ into the plane of rotation (e.g. 
$\vecN{u}_\bot\ds = \vecN{u} - (\vecN{u}\cdot\vecN{x}) \vecN{x}$).
The rotation angle is then defined via
\begin{equation}
	\psi = \text{atan2}(\sin\psi, \cos\psi)
		= \text{atan2}\left(\text{sign}(a)\abs{\vecN{u}_\bot\ds \times \vecN{v}_\bot\ds},\; 
		\vecN{u}_\bot\ds \cdot \vecN{v}_\bot\ds\right)\,.
\end{equation}
It is best to use $\text{atan2}$ because it is more precise
for angles near $0$, $\pi/2$ and $\pi$. Note that we have to 
inject the \emph{sign} of $a$ into $\sin(\psi)$, 
because when $a<0$, the RH rotation becomes larger than 
$\pi$, so we must instead uses a \emph{negative} RH rotation.

%~ An effective implementation should keep in mind that 
%~ many of these calculations are redundant (or terms cancel).
%~ These operations are numerically stable, provided that $\vec{u}$ is 
%~ not exactly parallel/anti-parallel, a condition that can be checked 
%~ before constructing $\vec{x}$.

I have tested an implementation of this algorithm and it works quite well
(it is both length and angle preserving). 
I have additionally validated that rotating once about $\ax$ gives 
the same result as the two-step composite rotation.
%~ The main problem is a slight change in magnitude when $\vec{u}$ and $\vec{v}$
%~ are mostly anti-parallel. This can be fixed with a
%~ magnitude preserving step that adds a 15\% overhead.
\end{document}
